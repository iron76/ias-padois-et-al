
\documentclass[12pt,a4paper]{article}
\usepackage{geometry}
\geometry{left=1.5cm,right=1.5cm,top=1.5cm,bottom=1.5cm}
\usepackage{graphics} % for pdf, bitmapped graphics files
\usepackage{epsfig} % for postscript graphics files
%\usepackage{mathptmx} % assumes new font selection scheme installed
%\usepackage{times} % assumes new font selection scheme installed
\usepackage{amsmath} % assumes amsmath package installed
\usepackage{amssymb}  % assumes amsmath package installed
\usepackage[nospace,sort]{cite}
\usepackage{mathtools}
\usepackage{graphicx}
\usepackage{subfig}
\usepackage{array}
\usepackage{enumitem}
\usepackage{longtable}
\newcommand{\vect}[1]{\mbox{\boldmath${#1}$}}%$

\title{Authors' Response}
\author{}
% Remove command to get current date 
\date{}
\begin{document}



\begin{flushleft}
Dear Guest Editor,
\end{flushleft}

First of all, the authors would like to thank you and the reviewers for your very pertinent and helpful comments in your letter dated December 20, 2015.\\
We apologize for having submitted a initial version where the figures numbering and citations where more than a mess and we really appreciate the level of details with which the reviewers have examined our work.\\
We have fully revised the paper according to your suggestions. We believe that the paper has definitely gained in clarity. A response to all the comments that were made by the reviewers is presented hereafter.


\section{Responses to the comments made by the guest editor}

\hspace{-0.5cm}
\begin{tabular}[!h]{|p{0.3cm}|p{8.5cm}||p{9cm}|}
\hline
$\sharp$ & Comments from the guest editor & Responses to the guest editor comments\tabularnewline  \hline \hline
1 & One reviewer said that the authors should more clearly state what
is the description of the state of the art and what are the novel
contributions made in the paper, since the two parts are mixed together.
As Co-Editors we  believe that it is important that the paper keeps
a general view on the state of the art. So we prefer that the paper
devotes a substantial part of its content to a general view, according
to the Call of NRF-IAS Workshop, since overview papers are very attractive
for the RAS Journal readers. Authors may keep the organization of
the submitted paper as is, but need to reorganize the introduction
and be more precise in separating the two aspects (State of the art,
and CoDyCo project). & While constructed on the base of the existing literature and state-of-the-art
 in the domain, the « roadmap beyond state-of-the-art » subsection
clearly offers an original view on the problem and there is, from
the authors point of view, no ambiguity on the fact that this view
(and the way the project was structured based on this initial work)
is a real contribution of the CoDyCo project. Then the « state-of-the-art »
section leaves no ambiguity in terms of what was clearly existing
before CoDyCo.\tabularnewline \hline
2 & Section 3 is very long, while section 4 is very short.  & The sections have been reorganised as suggested by reviewer 2: human-centered
findings and robot-centred findings have been separated into three sections
and section 4 has been merged in the control-related section.\tabularnewline \hline
3 & Additionally, as suggested by one reviewer the authors should enhance
the conclusions of the paper, including some material on expected
future work in the field and how their project could stimulate further
research. & The conclusion has been enhanced including a summary related to the
current work and future research directions.\tabularnewline \hline
4 & Finally, the authors should be very clear in specifying which parts
of the work have already been published and where, and what is new. & When some results presented in this paper have already been published,
an explicit reference to the corresponding paper is made.\tabularnewline \hline
5 & The reviewers comments contain many corrections of mismatching figures,
missing citations, and other errors. Please check the paper carefully. & These comments have all been accounted for.\tabularnewline \hline
6 & Possibly try to include some more recent results from the advancement
of Your CoDyCo project since your initial submission. & Some of the references have been updated in that sense.\tabularnewline \hline
\end{tabular}


\section{Responses to the comments made by the reviewer 1}

\hspace{-0.5cm}
\begin{tabular}[!h]{|p{0.3cm}|p{8.5cm}||p{9cm}|}
\hline
$\sharp$ & Comments from reviewer 1 & Responses to the comments of reviewer 1\tabularnewline  \hline \hline		
7 &	The paper also provides a review of the state of the art, but this is strongly biased toward the CoDyCo project. The scope of the paper should be clarified in the paper’s title, which should include a reference to the project.	&	A very  explicit reference to the CoDyCo project has been added in the title of the paper.\tabularnewline \hline
8 & The paper is well written, however the authors are suggested to re-read their paper before submission. Most of the references to the figures are wrong, and many of the links to external material (videos, etc.) are missing. Even some references (e.g. in Sections 2.2, 2.4, 3.2.2) are missing.	&	The videos and links have been checked the provided ones are functioning. The figures have been properly rearranged and missing references have been fixed. \tabularnewline \hline
9 & Advantages and the reasoning underlying the classification presented in Section 1 should be clarified. Alternatively, the section may be removed altogether. Indeed, the state of the art overview presented in Section 2 is not organized according to the classification presented in Section 1, and Section 1 does not seem to provide additional value to the paper in its current form.	&	Section 1 provides the general motivation behind the CoDyCo project and provides an original prism through which the overall topic of ``Whole-body multi-contact motion in Humans and Humanoids'' can be viewed. An additional paragraph has been added to the introduction to better describe the content of the paper and link it properly to the content of section 1. \tabularnewline \hline
10 &	A listing, and possible categorization of the material that follows, should instead be provided toward the beginning of the paper. In fact, the single sections present very heterogeneous material, and an overview of the relationship between them is what could make the paper worth, since the works themselves are all presented elsewhere, in more specific papers.	&	Cf. previous answer. \tabularnewline \hline
11	& The conclusions should be extended with the limitations of the work. Indeed, readers of this overview paper may be interested in knowing what kind of difficulties emerge, and which problems are still open	&	The conclusion has been enhanced including a summary related to the current work and future research directions. \tabularnewline \hline
12	& References to figure’s colors in the descriptions should be removed, and instead different line styles should be used (e.g. references to  ``blue '' for Fig. 4 or to the traces colors in Fig.18). Both black and white printing and color blind readers will be impaired by the current figures	&	Unfortunately it has not been possible due to time constraints to modify all figures in such a way.\tabularnewline \hline
\end{tabular}

\section{Responses to the comments made by the reviewer 2}

\hspace{-0.5cm}
\begin{longtable}[!h]{|p{0.3cm}|p{8.5cm}||p{9cm}|}
\hline
$\sharp$ & Comments from reviewer 2 & Responses to the comments of reviewer 2\tabularnewline  \hline \hline		
13 & The subject addressed in the paper is worthy of investigation however the clearness of the paper should be improved before its publication. Indeed, the paper is very long and rich of information, however, due to the lack of some details and a lot of confusion with the figures, it is difficult to follow. & The figures have been properly rearranged and some details have been added in accordance to the reviewer comments.\tabularnewline \hline
14 & Section  ``Introduction ''. It would be nice to add a paragraph the describe the organization of the paper. & An additional paragraph has been added to the introduction to better describe the content of the paper and link it properly to the content of section 1. \tabularnewline \hline
15 &	 Section  ``Some advances w.r.t. the state-of-the-art.... ''. In my opinion this section is very long and I would divide it into two sections, one describing the findings obtained studying humans and the other describing the results achieved with the robot. & The sections have been reorganised as suggested by reviewer 2: human-centred findings and robot-centred findings have been separated into three sections and section 4 has been merged in the control-related section.\tabularnewline \hline
16 & 	Section  ``Conclusions ''. After a so long paper, it is quite frustrating read a so short conclusion. I would resume the main findings I would say something more about the future work. & The conclusion has been enhanced including a summary related to the current work and future research directions.\tabularnewline \hline
17 & Page 2, column 1, second paragraph:  ``in the following figure '' could be replaced with  ``in Fig. 1 '' because otherwise Fig. 1 will not be referenced. & A proper reference to Figure 1 has been added.\tabularnewline \hline
18	& Page 2, column 2, first paragraph: I think that  ``Figure 1 '' should be replaced with  ``Figure 2 '' & The reference has been updated.\tabularnewline \hline
19	& Page 3, column 1, line 5:  ``Figure 1 '' $\rightarrow$  ``Figure 2 '' & The reference has been updated.\tabularnewline \hline
20	& Page 3, column 1, 2nd to last line:  ``Figure 2 ''$\rightarrow$  ``Figure 3 '' & The reference has been updated.\tabularnewline \hline
21	& Page 3, column 2, 3rd to last line:  ``Figure 3 ''$\rightarrow$  ``Figure 4 '' & The reference has been updated.\tabularnewline \hline
22	& Page 7, column 1:  ``Figure 4 '' $\rightarrow$  ``Figure 5 '' (three times). & The reference has been updated.\tabularnewline \hline
23	& Page 7, column 2:  ``Figure 5 '' $\rightarrow$  ``Figure 6 '' (twice) & The reference has been updated.\tabularnewline \hline
24	& Page 8:  ``Figure 6 ''$\rightarrow$ ''Figure 7 ''   ``Figure 7 ''$\rightarrow$ ''Figure 8 ''  ``Figure 8 ''$\rightarrow$ ''Figure 9 '' & The reference has been updated.\tabularnewline \hline
25	& Page 9:  ``Figure 9 ''$\rightarrow$ ''Figure 10 '' (twice)   ``Figure 10 ''$\rightarrow$ ''Figure 11 '' & The reference has been updated.\tabularnewline \hline
26	& Page 10:  ``Figure11 '' $\rightarrow$  ``Figure 12 '' & The reference has been updated.\tabularnewline \hline
27	& Page 10:  ``Figure 12 ''$\rightarrow$  ``Figure 13 '' & The reference has been updated.\tabularnewline \hline
28	& Page 10:  ``Figure 14 ''$\rightarrow$  ``Figure 13 '' ? & A figure depicting the effect of the repetition of sessions one the handle forces has been added.\tabularnewline \hline
29	& Page 11:  ``Figure 15 '' $\rightarrow$  ``Figure 14 '' (three times) & The reference has been updated.\tabularnewline \hline
30	& Page 11, column 2, last line:  ``Figure 16 '' $\rightarrow$  ``Figure 15 '' & The reference has been updated.\tabularnewline \hline
31	& Page 12, column 2, last line:  ``Figure 17 '' $\rightarrow$  ``Figure 16 '' & The reference has been updated.\tabularnewline \hline
32	& Page 13, column 2. :  ``Figure 18 ''$\rightarrow$  ``Figure 17 '' & The reference has been updated.\tabularnewline \hline
33	& Page 13, column 2. :  ``Figure 19 ''$\rightarrow$  ``Figure 18 '' & The reference has been updated.\tabularnewline \hline
34	& Page 14, column 2. :  ``Figure 20 ''$\rightarrow$  ``Figure 19 '' & The reference has been updated.\tabularnewline \hline
35	& Page 14, column 2. : Not sure if the reference to Figure 21 is right. Maybe it refers to Fig 20. The description here should be improved. One between Fig 20 and Fig 21 seems to be not referenced. & A proper reference to Figure 21 has been added.\tabularnewline \hline
36	& Page 15, column 2:  ``Figure 23 ''$\rightarrow$ ''Figure 22 '' & The reference has been updated.\tabularnewline \hline
37	& Page 16, column 2:  ``Figure 24 ''$\rightarrow$  ``Figure 23 '' & The reference has been updated.\tabularnewline \hline
38	& Page 17, column 2:  ``Figure 25 ''$\rightarrow$  ``Figure 24 '' & The reference has been updated.\tabularnewline \hline
39	& Page 17, column 2:  ``Figure 26 ''$\rightarrow$  ``Figure 25 '' (twice) & The reference has been updated.\tabularnewline \hline
40	& Page 19. Figures 26 and 27 are not referenced & These figures are now referenced.\tabularnewline \hline
41	& Page 4, column 2 $\rightarrow$ contact forces of complex manipulators and humanoids [?] & The correction has been made.\tabularnewline \hline
42	& Page 6, section  ``Model Learning ''$\rightarrow$ SVM and Neural Network[?] & The correction has been made.\tabularnewline \hline
43	& Page 6, section  ``Operational space control learning '' $\rightarrow$ iCub[?] ... compliance [?] & The correction has been made.\tabularnewline \hline
44	& Page 7, end of column 1 and column 2. Please add a reference to about the 3D dynamic model. & A reference to OpenSim has been added.\tabularnewline \hline
45	& Page 7, column 2,  ``Metric for postural ... '' subsection. Which joints are involved in the metric? Why? Explain better this part, please. & In order to clarify this concept, it has been clarified that all the actuated joints are accounted for in the computation of the metric for postural stability.\tabularnewline \hline
46	& Page 13, column 2, last paragraph $\rightarrow$ ... torque estimation [?] & The correction has been made.\tabularnewline \hline
47	& Page 17, column 1, first line $\rightarrow$ CMA-ES[?] & The correction has been made.\tabularnewline \hline
48	& Page 2, column 2, first paragraph:  ``genraly '' $\rightarrow$  ``generally '' & The correction has been made.\tabularnewline \hline
49	& Figure 4:  `` the metric space ''$\rightarrow$  ``The metric space `` (to be coherent in the notation) & The correction has been made.\tabularnewline \hline
50	& Page 5, column 2, Section 2.3: QP is not defined (Quadratic Programming?) & The acronym definition has been added.\tabularnewline \hline
51	& Page 5, column 2, Secrtion 2.3:  ``complexity and uncertainty '' $\rightarrow$ check the quotation marks & The correction has been made.\tabularnewline \hline
52	& Figure 5:  ``perturbation. the subjects '' $\rightarrow$  ``perturbation. The subjects '' & The correction has been made.\tabularnewline \hline
53	& Figure 5, left panel: I would insert the labels  ``posterior '' and  ``anterior '' near the arrows, and the labels  ``low '', ''mid '', ''long '', ''high '' near the sketches. & Additional labels has been added based on the suggestions of the reviewer. A reference to a more focused paper has also been provided.\tabularnewline \hline
54	& Figure 7 and 8: please, insert the labels of the z-axes & Proper labels have been added to the corresponding graphs.\tabularnewline \hline
55	& Figure 13: what does the symbol  ``¡ '' stand for? Perhaps is  ``< '' & The symbol has been corrected.\tabularnewline \hline
56	& Figure 16:  ``left '' $\rightarrow$  ``top ''   ``right ''$\rightarrow$ ``bottom '' & The correction has been made.\tabularnewline \hline
57	& Page 13, column 1: MPC is not defined & The acronym definition has been added.\tabularnewline \hline
58	& Page 13, column 1: ZPM is not defined (zero moment point?) & The acronym definition and a reference have been added.\tabularnewline \hline
59	& Figure 22: the symbols  ``plus minus '' is inverted and looks strange. & The symbol has been corrected. \tabularnewline \hline
60	& Page 16, column 1, last line: after  ``tasks '' a dot misses & The correction has been made. \tabularnewline \hline
61	& Figure 26. after  ``wrench '' a dot misses & The correction has been made. \tabularnewline \hline
\end{longtable}


We hope the manuscript has been improved satisfactorily and that it will be accepted for publication in the special issue on New Research Frontiers for Intelligent Autonomous Systems of Robotics and Autonomous Systems.\\

Sincerely,

Vincent Padois, Serena Ivaldi, Jan Babi\v{c}, Michael Mistry, Jan Peters and Francesco Nori

\end{document}
