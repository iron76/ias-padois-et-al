%% 
%% Copyright 2007, 2008, 2009 Elsevier Ltd
%% 
%% This file is part of the 'Elsarticle Bundle'.
%% ---------------------------------------------
%% 
%% It may be distributed under the conditions of the LaTeX Project Public
%% License, either version 1.2 of this license or (at your option) any
%% later version.  The latest version of this license is in
%%    http://www.latex-project.org/lppl.txt
%% and version 1.2 or later is part of all distributions of LaTeX
%% version 1999/12/01 or later.
%% 
%% The list of all files belonging to the 'Elsarticle Bundle' is
%% given in the file `manifest.txt'.
%% 

%% Template article for Elsevier's document class `elsarticle'
%% with numbered style bibliographic references
%% SP 2008/03/01

\documentclass[final,5p,twocolumn]{elsarticle}

%% Use the option review to obtain double line spacing
%% \documentclass[authoryear,preprint,review,12pt]{elsarticle}

%% Use the options 1p,twocolumn; 3p; 3p,twocolumn; 5p; or 5p,twocolumn
%% for a journal layout:
%% \documentclass[final,1p,times]{elsarticle}
%% \documentclass[final,1p,times,twocolumn]{elsarticle}
%% \documentclass[final,3p,times]{elsarticle}
%% \documentclass[final,3p,times,twocolumn]{elsarticle}
%% \documentclass[final,5p,times]{elsarticle}
%% \documentclass[final,5p,times,twocolumn]{elsarticle}

%% For including figures, graphicx.sty has been loaded in
%% elsarticle.cls. If you prefer to use the old commands
%% please give \usepackage{epsfig}

%% The amssymb package provides various useful mathematical symbols
\usepackage{amssymb}
\usepackage{xspace}
\usepackage{verbatim}
\usepackage[usenames]{color}
\usepackage[usenames,dvipsnames,table]{xcolor}
\usepackage{url}
\usepackage{array}
\usepackage{color}
\usepackage{longtable}
\usepackage{wrapfig}
\usepackage{tikz}
\usetikzlibrary{graphs}
\usetikzlibrary{backgrounds}
\usepackage{subcaption}

% \usepackage[pdftex,dvips]{graphicx}


%% The amsthm package provides extended theorem environments
%% \usepackage{amsthm}

%% The lineno packages adds line numbers. Start line numbering with
%% \begin{linenumbers}, end it with \end{linenumbers}. Or switch it on
%% for the whole article with \linenumbers.
%% \usepackage{lineno}

%% \journal{Nuclear Physics B}

\begin{document}

\begin{frontmatter}

%% Title, authors and addresses

%% use the tnoteref command within \title for footnotes;
%% use the tnotetext command for theassociated footnote;
%% use the fnref command within \author or \address for footnotes;
%% use the fntext command for theassociated footnote;
%% use the corref command within \author for corresponding author footnotes;
%% use the cortext command for theassociated footnote;
%% use the ead command for the email address,
%% and the form \ead[url] for the home page:
%% \title{Title\tnoteref{label1}}
%% \tnotetext[label1]{}
%% \author{Name\corref{cor1}\fnref{label2}}
%% \ead{email address}
%% \ead[url]{home page}
%% \fntext[label2]{}
%% \cortext[cor1]{}
%% \address{Address\fnref{label3}}
%% \fntext[label3]{}

\title{Whole-body multi-contact motion in Humans and Humanoids}

%% use optional labels to link authors explicitly to addresses:
%% \author[label1,label2]{}
%% \address[label1]{}
%% \address[label2]{}

\author{Vincent Padois}
\address{Sorbonne Universit�s, UPMC Univ. Paris 06 and CNRS Institut des Syst�mes Intelligents et de Robotique F-75005, Paris, France. Email: vincent.padois@upmc.fr }

\author{Serena Ivaldi}
\address{French Institute for Research in Computer Science and Automation (INRIA) Nancy Grand-Est, France. Email: serena.ivaldi@inria.fr}

\author{Jan Babic}
\address{Faculty of Electrical Engineering Josef Stefan Institute, Slovenia Email: jan.babic@ijs.si}

\author{Michael Mistry }
\address{University of Birmingham, UK Email: m.n.mistry@bham.ac.uk}

\author{Jan Peters}
\address{Max Planck Institute for Intelligent Systems and TU Darmstadt, Germany
Email: mail@jan-peters.net}

\author{Francesco Nori}
\address{Robotics, Brain and Cognitive Science Department Istituto Italiano di Tecnologia. Email: francesco.nori@iit.it}


\begin{abstract}
Traditional industrial applications involve robots with limited mobility. Consequently, interaction (e.g. manipulation) was treated separately from whole-body posture (e.g. balancing), assuming the robot firmly connected to the ground. Foreseen applications involve robots with augmented autonomy and physical mobility. Within this novel context, physical interaction influences stability and balance. To allow robots to surpass barriers between interaction and posture control, forthcoming robotic research needs to investigate the principles governing whole-body motion and coordination with contact dynamics. There is a need to investigate the principles of motion and coordination of physical interaction, including the aspects related to unpredictability. Recent developments in compliant actuation and touch sensing allow safe and robust physical interaction from unexpected contact including humans. The next advancement for cognitive robots, however, is the ability not only to cope with unpredictable contact, but also to exploit predictable contact in ways that will assist in goal achievement. Last but not least, theoretical results needs to be validated in real-world scenarios with humanoid robots engaged in whole-body goal-directed tasks. Robots should be capable of exploiting rigid supportive contacts, learning to compensate for compliant contacts, and utilising assistive physical interaction from humans.
\end{abstract}

\begin{keyword}
whole-body \sep control \sep free-floating \sep interaction \sep contacts \sep compliance.
%% keywords here, in the form: keyword  keyword

%% PACS codes here, in the form: \PACS code \sep code

%% MSC codes here, in the form: \MSC code \sep code
%% or \MSC[2008] code \sep code (2000 is the default)

\end{keyword}

\end{frontmatter}

%% \linenumbers

%% main text
\section{Introduction}
\label{sec:intro}

For cognitive agents, such as humanoid robots, to persist and act in natural human environments, contact and physical interaction become necessary and unavoidable. Everyday tasks involve making and breaking contact, among all areas of the body, whether the contacts are accidental disturbances or intentional support for dynamic movement. Critically, robots should be robust enough to cope with unpredictable contact, via safe control mechanisms and compliance.  Moreover, cognitive goal directed robots need the ability to exploit predictable contact, to aid in goal achievement, as well as learn dynamics of contact in order to generalise to novel tasks and domains.

Physical interaction has been studied in robotics, extensively under the umbrella of manipulation. For historical reasons, these studies have assumed a fixed-base as cur-rent industrial applications do not necessitate extended mobility. Foreseen robotic applications will demand an increasing level of autonomy, including physical mobility. These applications call for extending studies on interaction to cases where the robot has a mobile-base. Remarkably and differently from the fixed-base case, inter-action in these situations may compromise system balance, and goal directed actions require proper whole-body coordination and use of contact. However, the principles governing whole-body coordination in humans are far from being understood and implementations on complex systems, such as humanoids, are missing, especially besides walking.

Within this context one of the major challenges of robotic research is to advance the current control and cognitive understanding about robust, goal-directed whole-body motion execution with multiple contacts. Remarkably, focus should be posed on complex systems, such as humans and humanoids. In a crescendo of complexity, as illustrated in the following figure, current state of the art (state-of-art 1 and 2) should be advanced to address more complex scenarios (challenges 1 and 2).
State-of-art 1: balancing with multiple rigid contacts. The robot is standing and balancing with its hands supported by a rigid table in front of its body. However, the table is too fragile, and unexpectedly breaks. A contact state change is sensed, and the robot's control architecture automatically adjusts posture control parameters to maintain balance in light of the reduced support. The unexpected breaking of contact makes it more challenging.

State-of-art 2: goal directed actions involving contacts. The robot is standing with its hands at its side, and intends to reach for an object on a table in front.  The robot recognises that the distance is sufficiently far away, and the task cannot be achieved without compromising balance.  The robot decides to initiate a new contact with its left hand on the table, providing sufficient support for reaching the object with its right hand.
Challenge 1: learning non-rigid contacts. The robot sits down on a chair with a soft cushion, however the cushion has a particular stiffness quality not experienced before. The robot tries to reach for an object on a table, but it fails as it did not adequately compensate for the unexpected dynamics of the soft cushion.  After a few attempts, the robot adapts its model of the contact interaction, and is able to infer new control action to successfully reach the goal.

Challenge 2: human assistive contacts. The robot is seated in a chair, and a per-son comes to assist the robot to stand. He/she grabs both hands of the robot and starts pulling upwards.  The robot senses the new contact, and recognising from the interaction force that it is an external agent, allows its arms to be compliant.  When the force becomes sufficient to enable standing, the robot recognises the intended action and stiffens its arms while pushing its legs to rise from the chair. Finally once standing, but still in contact with the human, the robot returns compliance to its arms to allow for safe interaction while retaining overall control of its posture.

\begin{figure}[!ht]
\centering
\includegraphics[width=0.7\linewidth]{./images/scenarios.png}
%\label{fig:subfig2}
\label{fig:scenarios}
\end{figure}

Present day robots are still far from the human capabilities in exploiting predictable events and in coping with uncertainty. The gap between humans and robots is particularly apparent when in tasks involving unstructured physical interaction with the environment or other agents. Recent behavioural experiments yielded a new perspective on modelling the way humans deal with both predictable and unpredictable motor control tasks. In early experiments, it has been shown \cite{Shadmehr1994a} that humans learn and adapt internal dynamical models of their own arm in interaction with the environment. Such internal models appear to be crucial in predicting how muscle activations pro-duce hand movements and therefore may play an essential predictive role in movement planning. However, Burdet et al. \cite{Burdet2001} have shown that when prediction is not a viable strategy, humans can rely on arm compliance regulation (by means of muscle co-activation) to cope with the unpredictability that naturally arises from feedback delays when performing arm-reaching movements in unstable environments. Basic research and robotics technology are ready to extend such insights from single limb movements to whole-body interaction and the validation of these models appears feasible. In contrast to manipulation scenarios with static base robot systems dynamic whole-body interaction concerns the analysis of phenomena at a higher scale (bigger interaction forces, bigger muscle activations, etc.). whole-body compliance regulation with force/impedance control is not only favoured by current theoretical progress and available technologies, but may actually be ready for wide-spread use instead of being limited to just a few prototypes.

\begin{figure}
\centering
\includegraphics[width=\linewidth]{./images/classification1.png}
%\label{fig:subfig2}
\caption{Classification of whole-body tasks based on external-compliance. The complexity increases from top to bottom, i.e., with the need of exploiting the compliance of the contacts.}
\label{fig:classification1}
\end{figure}

\subsection{Roadmap beyond state-of-the-art}

With reference to Fig.~\ref{fig:classification1} and following, we propose a classification that relies on the well-known concept of compliance (or the inverse concept of stiffness), to be understood as the force-displacement characteristic of a contact. Interaction scenarios can be classified by quantitatively measuring two essential components of contacts: external and internal compliance (internal here refers to the agent or ``the self''). The first scenarios classification (Fig.~\ref{fig:classification1}) is based on the external-compliance; it includes scenarios that involve non-compliant (rigid) external contacts and scenarios with compliant external contacts. This second category is extremely wide in consideration of the multitude of possible compliant behaviours that can be experienced: from the linear force-displacement characteristic of a linear spring to the complex non-linear characteristic of a pillow. Scenarios within this category practically overlap with the first category but rigid contacts are replaced by non-rigid contacts. In these two categories the agent (or ``the self'', represented with a human silhouette) is always interacting with inanimate objects (the external contacts: a chair, a sofa, the floor, etc.). In the last category, ``the self'' and ``the other'' are both humans. In these scenarios the external-compliance is not a well-defined relationship between force and dis-placement but depends on the active intention of ``the other''.
External-compliance is only one side of the interaction, and the agent has limited control over it. The other side of the interaction is what we call the ``self'' (internal) compliance, which is instead fully under control of the cognitive agent. Self-compliance needs to be adapted to the environment compliance and the ability to actively regulate the internal compliance has been only recently implemented on multi-degrees-of-freedom robots. The self-compliance regulation represents the pro-active and cognitive component of the interaction and therefore gives the robot an enhanced degree of autonomy to be exploited in handling situations not anticipated at design time. In this sense, the self-compliance level and actuation range can be used to classify different scenarios as shown by Fig.~\ref{fig:classification2}. At the very first level of this classification we consider scenarios that do not require significant self-compliance regulation as they typically involve dynamically stable situations. Such situations involve for example dynamically stable tasks, which substantially require direct control of stable postures. The second level of the classification includes tasks that re-quire a certain level of active compliance either to stabilise unstable systems (e.g. balancing) or to compensate for unpredictable interaction characteristics (e.g. standing hand in hand with another agent). Finally at the highest level of this classification we consider highly complex tasks characterised by strong requirements in terms of ``self''-compliance planning and regulation.

\begin{figure}
\centering
\includegraphics[width=\linewidth]{./images/classification2.png}
%\label{fig:subfig2}
\caption{Classification of whole-body tasks according to an increasing self-compliance level and actuation range.}
\label{fig:classification2}
\end{figure}

External and self-compliance are two fundamental aspects of any interaction. It is therefore crucial to understand how these two concepts become intertwined once con-tacts are established. We will introduce the concept of contact-compliance, which corresponds to the overall compliance obtained once the external and the self-compliance become coupled with the contact establishment. A contact can be seen as the serial connection of two compliances, one representing the external-compliance, the other representing the self-compliance. The compliance of a serial interconnection is simply the linear sum of the individual compliances. Roughly speaking, the con-tact-compliance does not significantly change when the external and self-compliance are changed simultaneously by an equal and opposite quantity. No advancement can be associated to situations which correspond to augmenting the self-compliance at the cost of diminishing the external-compliance or vice versa, as in these situations the overall contact-compliance does not change. This fundamental procedural principle is well sketched in Fig.~\ref{fig:classification3}. The horizontal axis sorts possible scenarios according to a progressively increasing external-compliance level. The vertical axis instead orders the same scenarios by means of increasing self-compliance levels and actuation ranges: tasks involving minimal self-compliance regulation or low levels of compliance are shown at the bottom; tasks involving wide self-compliance regulation ranges including high compliance levels are at the top. The grey-colour-valued function shown in the space defined by these two axes is a qualitative evaluation of progress beyond the state of the art: dark grey is the state-of-the-art, increasing levels of blue represent step-by-step progress beyond state of the art. Progress in handling whole body con-tacts can be achieved only by simultaneously increasing the external and the self-compliance levels. Conversely, little advances are achieved when increasing the environmental compliance but reducing the active compliance component. Vice versa, a dual way to achieve little progress beyond the state of the art corresponds to scenarios that involve a strong self-compliance regulation but reduced external compliance.  

\begin{figure}[!ht]
\centering
\includegraphics[width=\linewidth]{./images/classification3.png}
\caption{the metric space to evaluate the progress work beyond the current state-of-the-art. Interaction is the inter-twined combination of two components, external and self-compliance, both contributing to the concept of contact-compliance. Whole-body scenarios should be evaluated in a metric space that takes into account how self and external-compliance contribute to contact-compliance. Contact-compliance is the sum of self and external compliance. Remarkably the major advances can be obtained by simultaneously advancing the external and the self-compliance requirements. The vertical axis represents both self-compliance levels and actuation ranges in consideration of the fact we are mainly interested in self-compliance regulation, actuation and control. The four proposed scenarios have increasing complexity with respect to current state-of-the-art. }
\label{fig:classification3}
\end{figure}

\section{State-of-the-art}

Technological state of the art. Among the recent achievements in the field of robotics, there are two major technological prerequisites that will play a fundamental role in enhancing whole-body motion capabilities: distributed force and touch sensing. Both technologies have been only recently integrated and used in (humanoid) robots, including the iCub \cite{MettaG._etal2010}.
Force control is a fundamental property for any autonomous agent in interaction with the environment. First attempts to regulate interaction forces relied on active force and compliance control schemes, typically coupled with custom mechanical designs such as the ones proposed in \cite{Salisbury1988} and \cite{Hayashi2007} , which were eventually implemented on successful commercial manipulators. Similar solutions have been eventually implemented on some humanoid platforms \cite{Cheng2006} \cite{Escande2010}, including the iCub \cite{Fumagalli2012} . Recent theoretical and technological advances have revealed the importance of intentionally introducing mechanical compliance in the design \cite{Pratt1995} and (even more recently) the necessity of actively regulating the actuator passive compliance \cite{Koganezawa2005} \cite{Tonietti2005} \cite{Migliore2005} . It is to be expected that within the next years robots such as iCub will be equipped with variably compliant actuation technologies at some (if not all) of the main joints \cite{Tsagarakis2009} \cite{Tsagarakis2011} .
Touch is another fundamental sensing capability for autonomous agents willing to interact with an unstructured environment or humans \cite{Argall2010a} . Whole-body distributed touch sensing has been only recently embedded on humanoid robots, but there already exist quite a few examples: Robovie-IV \cite{Mitsunaga2006} , RI-MAN \cite{Mukai2008} , Macra \cite{Hayashi2007}  and Meka \cite{Jain} , just to cite a few. The iCub already integrates a mature technology \cite{Schmitz2011} cover-ing the upper body, legs and feet soles. 
Finally, several open-source software libraries have been developed in the last years to support research in whole-body dynamics and contact simulation. Several dynamics simulators have been developed for robotics (see \cite{ivaldi2014} for a survey). The most interesting physics engines for our purposes are the ones with Featherstone-like forward dynamics calculation \cite{Featherstone2008} \cite{Yamane2006a} , built-in collision detection and stable numerical contact forces computations \cite{Nakaoka2007} . Among the kinematic and dynamic libraries it is worth citing HuMAnS , a toolbox for analysis and control of both human and humanoid motion, and iDyn, a generic software tool, extensively used in iCub to compute whole-body dynamics and reinforce these computations with measurements coming from sensors embedded in the robot \cite{Fumagalli2012}.
Human motor control state-of-the-art. Human whole-body motion control has been studied within tasks such as reaching on a supporting surface and sit-to-stand. These movements involve coordination of multiple joints, significant shift of the centre of mass, and control of equilibrium, either in static or dynamic conditions. These are skills learnt early in human childhood but also studied extensively in the context of motor disability, e.g. after neurological insults like stroke, or in the elderly with reduced muscle power, joint flexibility and sensory loss. However, almost nothing is yet known about when healthy subjects choose to make use of contacts with support surfaces. It has been shown that in standing posture, this contact provides augmented sensory information reducing sway \cite{Johannsen2012} , and how in some circumstances, non-weight bearing but informative ``light-touch'' between two standing subjects can cause coupling that leads to increased sway, emphasising that knowledge about the stability and compliance of the contact surface is vital.
Reach using supports. Human reaching with arm support has not been extensively studied. There is almost no literature on the issue of how humans use one hand to extend their reach space. For example, to lean forwards requires a shift of the trunk and a shift of the centre of mass \cite{Huang2011} . At some point it becomes advantageous to use a supporting surface, allowing a reduction in anticipatory postural adjustments and a simplified control strategy \cite{Slijper2000} . But the decisions about when to implement support using one arm, which will depend on the availability, reliability and compliance of a support surface are almost unstudied \cite{Krishnamoorthy2004} .
Sit-to-stand. The postural adjustments that contribute to a sit-to-stand action are well documented. The action requires a shift of centre of mass, development of momentum, and precisely timed hip and knee extension, combining with maintaining stability with ankle control. As motor ability lessens, e.g. in the elderly, compensatory foot placement with increased momentum generation using hip flexion and arm movement is often employed \cite{Janssen2002} . Support from the chair arm or from a cane \cite{Leung2009}  increases stability in the forward axis. Again, decisions about when the support surface would be used, depending on its stability and compliance, are unstudied. The effects of unstable foot support in the sit-to-stand action are studied in \cite{Huang2011} the authors suggests a clear trade-off between support surface stability and manoeuvrability, and argue that adapting to the added uncertainty could help individuals become more manoeuvrable. Finally, there is little work on how the sit-to-stand action changes with elastic support - this has been studied in locomotion and jumping \cite{Arampatzis2001} , but not in inter-actions with support surfaces.
Dimensionality reduction. Complex multi-joint movements call for control strategies that simplify and reduce degrees of freedom. There are various competing theories of how this can be achieved \cite{Latash2010} . Perhaps most relevant is the uncontrolled manifold hypothesis \cite{Scholz1999} that demonstrates that it is highly effective to allow some parameters to be uncontrolled, if task irrelevant, and to control only a fewer task relevant parameters. In \cite{Scholz1999} Scholz \& Schoner applied this to the sit-to-stand task, and show that the centre of mass in the forward axis is well controlled, head and hand position are less controlled, and vertical head position appears little controlled. How these behaviours change with support is an open question. Equally important is the issue of how high dimensionality whole body motion of human models can be reduced to extract principles of action applicable to robots with different geometries. These are implemented by muscle and joint synergies that reduce the functional degrees of freedom during a given action. There has been very effective use of principal or independent components analyses to capture such human whole body movement and reduce dimensionality (e.g. as in \cite{Forner-Cordero2005} ). Recent developments include extracting functional components, which treat joint-kinematics data as functions instead of as a series of independent samples, and are comparable across groups of subjects \cite{Coffey2011} .
Robot Motor Control state of the art. In complex scenarios, when the robot and the environment are assumed to be perfectly known, planning approaches explore the possible states of the robot (e.g. configurations of the robot in its environment) in probabilistic graph-like manners \cite{LaValle2006} to determine the sequence of commands to pro-vide to the robot to perform a certain action in free space \cite{Dalibard2009} \cite{Kuffner2001} or in complex contact situations \cite{Bouyarmane2009} \cite{Bouyarmane2011} . Such methods are usually computationally demanding and difficult to apply online. Conversely, when the global goal of the robot is relatively simple, the high-level planner can be almost disregarded because the goal to be achieved can ``easily'' be described a priori in terms of operational tasks \cite{Khatib1986} to be activated and combined. This falls into ``the simultaneous management of multiple operational objectives'', a well-known problem in model-based reactive control. The most popular method to deal with a set of objectives is a hierarchical framework, where operational tasks are typically prioritized in a ``stack'' \cite{Mansard2007} , which found sever-al applications to humanoids \cite{Sentis2010} \cite{Mistry2011} . QP solvers have recently gained popularity in humanoid robotic as they do not require the explicit inversion of any model of the system \cite{abe2007} \cite{colette2008} \cite{Escande2010} \cite{Salini2011} . This corpus of reactive methods mostly succeeds in over-coming the "complexity and uncertainty" factor thanks to the use of feedback. How-ever the proposed solutions are only locally optimal and the overall decision-making process cannot be addressed in the most general cases (i.e. without scripted scenarios). There is obviously a need for approaches where planning and reactive control are combined in a strongly intertwined way. This is not a simple problem: there are very few works where such a combination has actually been tested in a non ad hoc manner. The work of \cite{Alami1997} contributed to describe the necessary control architecture but did not propose any general control solution for such a combination to exist in practice. More recently \cite{PhilippsenRolandandNejatiNeginandSentis2009} introduced an architecture combining a whole body control level and a reactive symbolic planning, while \cite{yoshida2005} focused on dedicated mission-level planning methods for humanoids, coupled to task-level controllers. More recently, \cite{Salini2011} \cite{salini2011b} have proposed an architecture where sequences of operational tasks are generated on the fly based on a fuzzy-logic, rule-based decision engine. This approach, even though efficient in various specific applications, fails to scale-up as the number of required rules explodes with the growing complexity of the considered scenarios.
Learning state-of-the-art. Real-world environments are often hard to capture perfectly with physical models. The uncertainty in model predictions is important during controlled physical contacts between a (humanoid) robot system and its environment. Large errors either in the environmental model or in the task will lead to drastic failures and therefore need to be limited as much as possible by model adaptation. Hu-man-inhabited worlds will never allow perfect modelling and instead require that the system generalises the tasks in such a manner that they work in a wide variety of different uncertain scenarios where there is contact between robots and either humans or physical objects. Machine learning approaches are therefore needed. Particularly in whole-body motion they are necessary for the successful implementation of the control architecture, and its implementation and application to the real-worlds scenarios. However, off-the-shelf machine learning methods are concerned with static data sets and require massive amount of computations, often rendering real-time learning in-feasible. To date, a variety of robot learning approaches have been suggested. The most important being model learning, operational space control learning, learning of elementary tasks and hierarchical combinations of tasks, which are briefly evaluated hereinafter.
Model learning. High model accuracy and constant model adaptation may be key for low torque interaction during contact. Models of the robot dynamics have been learned by real-time regression, e.g., locally weighted projection regression \cite{Schaal2002a} and local Gaussian process regression \cite{Nguyen-tuong2010a} . Nevertheless, if any of these approaches would be given the data from a robot in contact with the environment, it would fit the model to this particular case, as the contact forces would just be treated as an additional nonlinearity. As a result, the model will not generalise to new contact models and instead it would be necessary to learn a new model for each type of contact.
Operational space control learning. Control in operational space has been approached both as a direct policy learning problem \cite{Peters2008a} as well as an indirect learning problem via forward models \cite{Salaun2010} . Here, the problem may be even more drastic as changing the contact formulation will alter the problem in its essence. As a result, an operational space control law may not transfer at all but rather become highly problematic under new circumstances. 
Learning of elementary tasks and hierarchical combinations of tasks. While learn-ing of contact-free elementary tasks by the combination of imitation learning and policy search \cite{Abbeel2005} \cite{Kober2010} is a well-explored topic, no general approaches to date can tackle the exact same problem and allow for different contact combination. Furthermore, learning of hierarchical elementary task combinations is still in its infancy. Several interesting approaches have been suggested \cite{Stulp2011a} \cite{Mulling2010} \cite{Muico2011} \cite{Daniel2012} in literature, relying on substantially different insights. Further exploration in this area is clearly needed, especially in unexplored multi-contact scenarios.
While all of these frameworks are well motivated in their domains, they have two major shortcomings from the viewpoint of whole-body motion control: they do not explicitly incorporate contact, and they do not leverage on the analytical robotics and control knowledge surrounding them.

\section{First year results}

\subsubsection{Work packages progress}

\paragraph*{WP1: toolbox for computing and controlling dynamics of whole-body movements with contacts (UB)}

WP1 objectives were achieved for the first year. In summary, the main accomplishments and impacts for the research community are as follows: 

\begin{itemize}

\item A survey of over 100 robotics researchers worldwide was conducted on use of simulator tools.  To date, there has been no such objective evaluation of the various simulation packages available for conducting robotics research. The output of such a survey will help researchers determine the best tool for their application and thus, is of great value to the community.  A report of survey results was released as part of deliverable D1.1 and was submitted for journal publication. 

\item A new iCub simulator to cope with multi-contact dynamics has been released. The new simulator can now interface with Gazebo, a widely used and actively developed simulation engine. The simulator has been documented as part of deliverable D1.1

\item A tool for automatic generation of URDF models of digital humans has been developed. As URDF is a standardized format for representing articulated rigid bodies, researchers  now have ease to create human kinematic and dynamic models for simulation and analysis. Documentation of the tool was released as part of D1.1. 

\item New methods for whole-body dynamics estimation have been developed, and submitted for publication. 

\item A new Gaussian process (GP) model was developed that is explicitly designed to deal with non-linearities induced through contacts with the environment. This work was submitted for publication. 
 
 \end{itemize}
 
\paragraph*{WP2: understanding and modelling human whole-body behaviours in physical interaction (JSI)}

After the first year of project, all planed objectives have been reached and one experimental modification has been implemented.

A thorough review was created with summary of the recent literature on human postural control and whole body motion in contact with environment. It includes relevant publications up to date and reviews the methods for evaluation of postural stability of bipedal systems beyond available reviews \cite{Mergner2007, Azevedo2007}. The review provides a solid bridge in methodologies and terminologies used by the project partners from multi-disciplinary backgrounds. Based on the overall objectives of the project and the specific objectives of WP2, experimental setup was created and procedures were defined. We obtained ethics committee approval for all project related human experiments (approved by National Medical Ethics Committee of Republic of Slovenia, reference number 112/06/13).

We performed an experimental study and examined functional role of supportive hand contact at different locations where balance of an individual was perturbed by translational perturbations of the support surface. We examined the effects of handle location, perturbation direction and perturbation intensity on the postural control and the forces generated in the handle. We found that an additional supportive hand contact significantly reduced the maximal displacement of the subject's centre of pressure (CoP) regardless of the position of the handle, direction of the perturbation and its intensity. This is in agreement with the accepted belief that an additional hand contact with support in general reduces the destabilizing effect of balance perturbation \cite{Maki1997, Bateni2005, Maki2006, Wing2011}. On the other hand, the position of the handle had no effects on the maximal CoP displacement. This supports the idea that maintaining postural stability is the task of the highest priority and that the central nervous system does whatever necessary to keep the body balanced \cite{Winter1995}. Our findings are in contrast with the recent findings of Sarraf et al. \cite{Sarraf2014}. We submitted a manuscript explaining the results of our study to Gait \& Posture Journal. The manuscript is under review after minor revision and is expected to be published by the end of 2014 \cite{Babic2014}.

Most of the human studies that examine postural control \cite{Horak1986, Henry1998, Dimitrova2004} (including our above mentioned study) utilize one-time support perturbations that unpredictably perturb the balance of an individual. During our experiments we noticed that human subjects reacted to all such perturbations regardless of how small or slow the perturbation was or what was the initial acceleration of the perturbation. Our conclusion was that these reactions are essentially protective reactions that do not necessarily have counterbalancing effects \cite{McIlroy1995, Corbeil2013}. Such reactions mask the real factors involved in human choice of contact utilization. We therefore altered the perturbation methods for our further experiments and designed continuous perturbations in a frequency band that corresponds with typical human motion during postural control \cite{Nawayseh2006}.


\paragraph*{WP3: control and optimization of whole-body motion in contact (UPMC)}

After one year of project, the level of achievement of the objectives in WP3 meets the expectations.

The whole-body control frameworks developed by J. Salini and A. Del Prete as part of their respective PhD thesis \cite{salini2012}, \cite{delprete2013} have been tested for simple rigid, multi-contact scenarios in the XDE \cite{XDE} and Gazebo \cite{Gazebo} physics simulators. These two simulators have been retained in WP1 (Deliverable 1.1) as modular simulation frameworks dedicated to the evaluation of the control strategies in CODYCO.

In the meantime, a novel "Generalized Smooth Hierarchical Control" algorithm has been developed \cite{liu2013}. It offers a rich way of describing and solving multi-task problems under constraints: both strict and soft tasks hierarchies can be enforced, tasks can be inserted and removed in a continuous manner and their priorities can be switched smoothly. It appears as a potential alternative to recent work in this domain \cite{escande2012}. Alternatively, TUD has worked on a Bayesian optimization framework dedicated to the bipedal locomotion gait optimization \cite{calandra2014}, \cite{calandra2014b}.

Regarding the exploration of the potential ways of coupling the local, reactive control level and the global, decision making one, several works have been initiated mostly related to the generation of "globally optimal" reference trajectories to be tracked reactively by the local controller. The contributions in this domain over the first year of project are mostly related to the work of A. Ibanez \cite{ibanez2013}, \cite{ibanez2014-icra} and \cite{ibanez2014-ark}. The distributed MPC approach developed in this work tackles the locomotion and balance problem from a new perspective that shares similarities with recent contributions such as \cite{mordatch2012} where an optimization framework enables an automated generation of rich contact behaviors, and \cite{ott2013} that combines a kinesthetic teaching task with an algorithm partially inspired by our approach to improve the balancing behavior during interactions. In the meantime, TUD investigated the interchange of forces during cooperative tasks between humans and robots \cite{berger2013}.

\paragraph*{WP4: adaptation, generalization and improvement of compliant control and tasks with contacts (TUD)}

The goal of WP4 is to endow the CoDyCo
humanoid robot control architecture with the
core abilities for the adaptation, generalization
and self-improvement of both control laws and
tasks that involve physical interaction with
humans, and the environment.
During the first year, developed a novel 
probabilistic movement primitive representation that
can be used for imitation learning and for superimposing multiple motor tasks. 
TUD also started to investigate to predict the partner's behavior in 
human robot interaction scenarios. 

\paragraph*{WP5: systems integration, standardization and evaluation on the iCub robot (IIT)}

The first year WP5 activities have concentrated on the first year validation scenario. A complete description of the scenario can be found in ``D5.1 Scientific report on validation scenario 1: balancing on multiple rigid contact points.'' which discusses the technical implementation of the first year validation scenario (see \url{https://github.com/robotology-playground/codyco-deliverables/tree/master/D5.1/pdf}). With respect to the state of the art the work progress represents an implementation of well established torques controlled whole-body control strategies. The integration of tactile feedback within the whole-body controller is a peculiarity of the implemented CoDyCo validation scenario and therefore represent a step forward with respect to the current state of the art. At the moment of writing the current deliverable the iCub tactile sensors cover the feet, the torso, the arms and the hands and the implemented validation scenario accounts for contacts at the hands and feet.

\paragraph*{WP6: management (IIT)}

The CoDyCo project started successfully. Management activities included the definition of an amendment procedure smoothly organized by the consortium and the project officer. A software repository (\url{https://github.com/robotology/codyco}) was set up using state of the art versioning tools (git) and social coding website (\url{https://github.com}). 

\paragraph*{WP7: dissemination and exploitation (IIT)}

Within WP7, CoDyCo first year achievement include: dissemination at relevant academic and industrial events; realization of a CoDyCo experiment database to disseminate robot and humans datasets. 

\paragraph{Work package 1 progress}

\subparagraph{Software architecture design and evaluation of available open-source software pertinent to the scope of the project. (T1.1)}

The explicit goal of T1.1 is for the consortium to agree on a specific software architecture with associated software tools whose specifications, dependencies and interconnections meet the requirements and needs for achieving the goals of the project.  To this end, the consortium met on 5th June 2013 at UPMC to discuss and agree on software interfaces, modules and architectures. The main outcomes from this meeting were: 
\begin{itemize}
\item IIT to develop plugins for Gazebo to interface with YARP. Gazebo chosen to be a replacement physics core for the iCubsim (see T1.2). 
\item UPMC to develop software using Orocos/XDE for whole body control and define generic interfaces for controllers, models, sensors, and actuation allowing for the communication of a C++ Orocos-based controller component with a robot using YARP or a YARP/Gazebo-based simulator.
\item  The consortium agreed on URDF as a unified modeling structure for defining and sharing descriptions of robots and human models. URDF is a standard XML format for representing the kinematic and dynamic description of a branched structure of articulated rigid bodies. 
\end{itemize}
The software architectural designs and specifications are to be documented as part of D1.2 and released at the end of year 2. 

\subparagraph{Simulator for whole-body motion with contacts (T1.2)}

The CoDyCo project requires a modular, component-based dynamics simulation software providing numerically stable, computationally efficient and physically consistent simulations of whole-body virtual human(oid) systems in contact with rigid or soft environments. To this end, in year one, a new iCub simulator has been released and documented as part of deliverable D1.1. In summary: 

\begin{itemize}
\item The previously existing iCub simulator needed an upgrade for more advanced applications including the multi-contact dynamics required for the CoDyCo project. The goal was to replace the physics core from ODE (Open Dynamics Engine) to one more suitable for articulated rigid body structures commonly used in robotics. To this end, Gazebo and XDE were chosen and evaluated as physics cores for the new iCub simulator. 

\item Partner UPMC led a survey of existing simulators for robotics\footnote{http://arxiv.org/abs/1402.7050}. In total 119 international robotics researchers responded to the survey.  

\item IIT contributed in CoDyCo with a joint activity with two other EU projects: Koroibot and WALK-MAN. The result of this collaboration is the development of a Gazebo plugin for exposing a YARP interface to the simulator. The plugin has been released with an open-source license and it is available on github (\url{https://github.com/robotology/gazebo_yarp_plugins}). At the moment of writing the current report, the plugin can be instantiated to control both COMAN (\url{https://github.com/EnricoMingo/iit-coman-ros-pkg}) and iCub (\url{https://github.com/robotology-playground/icub_gazebo}). This activity is related to a preliminary workshop publication \cite{Mingo2014}. 

\item Partner UPMC conducted a comparison between the XDE and Gazebo iCub simulators and a real iCub performing a leg free-falling task. In summary, in terms of predicted simulated outcomes, XDE and Gazebo are nearly numerically identical. However, both suffer with respect to accuracy, as the viscous friction models used are not able to accurately model the actual joint friction in the iCub. In conclusion, work in T1.4 (as well as WP3 and WP4) will need to address the issue of accurate friction modeling and estimation.

\item In order to provide a way to generate URDF models of digital mannequins independently from a specific simulator, JSI has developed a software for generating instances of a parametrized digital human (similar to the one present in XDE) as well as to edit the detailed parameters of an existing instance.  The new digital human URDF file generator is detailed in D1.1.
\end{itemize}

\subparagraph{Control library for flexible specification of task space dynamics of floating base manipulators. (T1.3)}

During the first year both IIT and UPMC contributed to the development of several software components for controlling the iCub whole-body behavior. The software has been structured around an abstraction layer called wholeBodyInterface, described in details within T3.2. This C++ abstraction layer is already used in a set of whole-body controllers implemented in simulink and available on github at the following address: \url{https://github.com/robotology/codyco/tree/master/src/simulink/controllers}. Within this context simulink is currently adopted as a fast designing tool for testing several controllers whose final implementation is foreseen in C++. Similarly, UPMC has started adopting the wholeBodyInterface within their own
controller framework based on XDE and ORCISI. Preliminary results are available here: \url{https://github.com/robotology/codyco/tree/master/src/tests}. 

\subparagraph{System dynamics estimation software. Extension to
environmental compliance estimation (T1.4)}

The goal of this task is to develop a software tool for on-line estimation of whole-body dynamics of the robot, as well as the compliance of contacts established between the robot and the environment. 

\begin{itemize}
\item Within T1.4, IIT contributed with two activities in year 1: developing a library for whole-body dynamics estimation and starting the activity of compliance estimation. Details of these activities can be found primarily in the associated scientific publications \cite{Traversaro2013, Traversaro2014, Fiorio2014}.

\item Also during year one, TUD started to investigate the learning of dynamic models with
discontinuities. A new Gaussian process (GP) model was developed
that is explicitly designed to deal with non-linearities
induced through contacts with the environment. An example of such non-linearities 
and the approximated model reconstruction is shown in Figure \ref{fig:example_discontinuities}.
We called the developed supervised learning method manifold GP (mGP), as it 
jointly learns a transformation of the data into a feature space, and a GP regression 
 from the feature space to observed space. In future work, this promising approach 
 will be applied to motor skill learning task on the iCub with multiple contacts. 
 A preprint of this work 
 was published this year [Calandra, R. and Peters, J. and Rasmussen, C. and Deisenroth,
M., 2014].
\end{itemize}

\begin{figure}
\centering
\includegraphics[width=0.7\linewidth]{./images/AAAI2014_0.pdf}
%\label{fig:subfig2}
 \caption{Illustration of a discontinuous function (black dots) that is approximated by three 
 model learning approaches. Classical Gaussian process regression methods (GP SE-ARD, and GP NN) 
 poorly reconstruct this function as they average over the ``jump'', which results in a high model variance. 
 TUD, however, demonstrated that by jointly learning a transformation of the data into a feature space, 
 and a GP regression model from the feature space to observed space, the non-linear function 
 can be reconstructed without large reconstruction errors. 
}
\label{fig:example_discontinuities}
\end{figure}

\subparagraph{Extension and enhancement of the iDyn library. (T1.5)}

The original iDyn library \url{https://github.com/robotology/icub-main/tree/master/src/libraries/iDyn} was designed assuming the robot in a fixed-base configuration. Within CoDyCo the library was redesigned in order to support floating base structures. The associated source code is available here \url{https://github.com/robotology/codyco/tree/master/src/libraries/iDynTree}. Within the YARP and iCub contexts, the libraries are used in the wholeBodyDynamics modules (\url{https://github.com/robotology/icub-main/tree/master/src/modules/wholeBodyDynamics} and \url{https://github.com/robotology/codyco/tree/master/src/modules/wholeBodyDynamicsTree} respectively) to compute simultaneously internal (joint torques) and external (contact) forces and torques. 

\paragraph{Work package 2 progress}

\subparagraph{Definition and design of experimental protocols (T2.1)}

The aim of the work in this task was to provide a solid multidisciplinary base for future research work within CoDyCo. We made a thorough review and summary of the recent relevant literature on human postural control and whole body motion in contact with environment (Delivery 2.1). The review examines postural control strategies without and with additional support contacts, types of perturbations that are commonly used to study neuromuscular functions involved in postural control and reviews the methods for stability evaluation of bipedal systems. The review is concluded with examination of stability metrics that can be applied for non-planar contacts. We plan to extend the review with methods of determination of inertial parameters of human/robot body and submit it for publication in a robotic journal by the end of 2014.

At JSI, we created an experimental setup to study human postural control and whole body motion in contact with environment. We implemented two state-of-the-art methods for perturbation of balance as shown on Figure \ref{fig:exp_protocol_W2} that will allow us to gain understanding how human brain deals with environment in the sense of supportive contacts. Using the same setup we can also validate all biomechanical findings on robotic systems by simply substituting the human subject with a robot. Besides, work has been undertaken to setup experiments also at UB. New equipment (the Moog Hapticmaster robot) was acquired and configured at both JSI and UB. Hapticmaster robot will be used in experiments with compliant and unpredictable contacts.

\begin{figure}
\centering
\includegraphics[width=0.8\hsize]{images/exp_setup_WP2.png}
\caption{Experimental setup to study human postural control and whole body motion in contact with environment. Front and back waist-pull motors together with the two force sensors located at the subject's waist allow real-time force perturbations of the subject while the Stewart platform can perturb the balance by either translational or rotational motion (or a combination of both) of the support surface. Force plates are used in combination with kinematical and electromiographical measurements (not on the figure) to study the adaptation of subjects to the given perturbations.}
\label{fig:exp_protocol_W2}
\end{figure}

\subparagraph{Design of models for human whole body motion in contact (T2.2)}

Work has begun on understanding how to derive simplified models of whole-body balance that will encapsulate the task relevant parameters of posture control with multiple contacts. By emulating situations when balance of an individual is challenged, we examined functional role of supportive hand contact at different locations where balance of an individual was perturbed by translational perturbations of the support surface. The experimental methods rested upon our work in Task 2.1 and are depicted on the left side of Figure \ref{fig:exp_paper_W2}.

\begin{figure}
\centering
\includegraphics[width=0.8\hsize]{images/exp1_JSI.png}
\caption{Examining functional role of supportive hand contact. The subjects were standing on a force plate mounted on top of the Stewart platform that generated translational perturbations. the subjects were holding the handle with a built-in force sensor in four different positions. Major results of the study are shown on the two diagrams on the right side.}
\label{fig:exp_paper_W2}
\end{figure}

We found that an additional supportive hand contact significantly reduced the maximal displacement of the subject's centre of pressure (CoP) regardless of the position of the handle and the type of the perturbation. On the other hand, the position of the handle had no effects on the maximal CoP displacement (top right diagram on Figure \ref{fig:exp_paper_W2}) which is against the previous belief that the quality of postural control depend on the location of the hand contact \cite{Sarraf2014} and supports the idea that maintaining postural stability is the task of the highest priority and that the central nervous system does whatever necessary to keep the body balanced \cite{Winter1995}. Specifically, subjects always generated the required hand force, no matter where the location of the handle was, to keep the body balanced to the same extent. To get a better understanding of the functional role of supportive hand contacts, we examined the handle forces exerted by the subjects during the perturbation. In contrast with the effects on CoP, we found significant effects of perturbation direction, perturbation intensity and handle position on the maximal force in the handle (bottom right diagram on Figure \ref{fig:exp_paper_W2}). A manuscript with the results of the work in T2.2 was submitted for publication in Gait \& Posture journal in December 2013 and is under review \cite{Babic2014}.

To properly model all these findings we developed a reduced dimensional (6-link, planar) model of a humanoid to be used as an inverse dynamics model for computing joint torques from human experimental data. The detailed analyses based on this model are under way at JSI and UB.

A major challenge of this task is understanding how to determine and measure stability when a human or humanoid robot is in a multi contact situation. The state-of-the-art in postural stability uses traditional metrics such as centre of pressure or zero moment point. However these planar metrics do not apply when there are multiple non-planar contacts. In addition to reviewing the current literature (Task 2.1), we have begun development of new methods for measuring stability margins when a human or humanoid robot has multiple non-planar contacts.

\subparagraph{Human contact choice and learning through physical interaction (T2.4)}

In order to understand how humans make contact choice decisions (e.g. whether or not to initiate a hand contact, and where to place the hand), we need an estimation of joint torques as well as a metric of stability in various multi-contact situations. Thus the work we have begun in Task 2.2 in terms of both simplified models of postural control and metrics of stability, also apply for Task 2.4.

To understand the factors involved in human choice of contact utilization, we performed a series of experiments where the subjects were standing still with arms hanging freely at the sides. The parallel platform induced a randomly timed series of perturbations of different accelerations, velocities and displacements. The aim of the experiments was to investigate what profile of support perturbation forces the human to make a supportive hand contact with environment and how human chooses the location of the hand contact with regard to the direction of the perturbation. Interestingly, we found that the subjects reacted to every perturbation no matter how small or slow the perturbation was or what was the initial acceleration of the perturbation. The reactions were manifested as muscle twitches of shoulder or as unspecific arm motions that were unrelated to the proximity of possible support objects. The reactions occurred also at the smallest perturbations when no actual correction of balance was needed. Our experiments showed that these reactions are essentially protective reactions rather than reactions that have counterbalancing effects \cite{McIlroy1995, Corbeil2013}.

These reactions mask the real factors involved in human choice of contact utilization. We therefore altered the perturbation methods for our further experiments and designed continuous random perturbations in a frequency band that corresponds with typical human motion during postural control \cite{Nawayseh2006}. By doing so we excluded the effect of surprise that evoked the reflex reactions of humans. This will hopefully allow us to uncover the factors involved in human choice of contact utilization.

\paragraph{Work package 3 progress}

The progress for each task are described hereafter.

\subparagraph{Reproducing existing control results in a simple case (T3.1)}
During year one, UPMC has achieved task T3.1 by creating a stand-alone C++ library encapsulating the whole-body controller developed in \cite{salini2012} so that it can be used by all partners in simulation or on (any) real humanoid robot. This version of the controller has been tested in rigid multi-contact scenarios in simulation (see Fig.~\ref{fig:xde}) and is currently adapted for tests on the iCub robot.

\begin{figure*}
\begin{center}
\includegraphics[width=0.3\hsize]{images/s5.png}
\includegraphics[width=0.3\hsize]{images/s6.png}
\includegraphics[width=0.3\hsize]{images/s4.png}

\includegraphics[width=0.3\hsize]{images/s3.png}
\includegraphics[width=0.3\hsize]{images/s2.png}
\includegraphics[width=0.3\hsize]{images/s1.png}
\end{center}
\caption{Screenshots of the validation scenario simulated in XDE.}
\label{fig:xde}
\end{figure*}

\subparagraph{Formulating the control problem (T3.2)}

The work performed during year one by UPMC to achieve T3.2 has led to the definition of what a task can be considered to be in the context of the reactive formulation of a multi-task whole body control problem. Among the different characteristics of a task (physical frame, task variable, forward model, desired target trajectory, local controller, priority), the notion of task priority has been largely modified with respect to the classical lexicographic task ordering met in the robotics literature and which is particularly appropriate for cascade resolution approaches such as the one recently proposed in \cite{escande2012}. A partial order has been defined such that task priorities can be described for any pair of task $i$ and $j$. This leads to a richer formulation which includes the original one but is also particularly appropriate for describing task insertion and removal processes as well as priority switching between tasks. Further more, this new prioritization paradigm provides a unique way of defining strict and soft hierarchies between tasks. Associated to this work, the notion  of generalized task projector has been introduced. Each task is associated to a projector which is built based on the tasks priorities. The interest of this projector is that it filters the joint space motion associated to a task so that all priorities are respected, being them soft or strict. Details regarding this work are provided in \cite{liu2013}.

Within this task also IIT contributed with the definition of a software abstraction layer, named wholeBodyInterface (\url{http://wiki.icub.org/codyco/dox/html/namespacewbi.html}). This software library defines the interface to access and control the robot  whole-body. Therefore the wholeBodyLibrary structures the control problem and their definition. Currently the library has been been implemented for both the iCub and the Gazebo iCub simulator.  

\subparagraph{Solving the local control problem (T3.3)}

Associated to the new task formulation proposed in T3.2, the control problem has been formulated by UPMC as an LQP which can be solved by any convex optimization solver dealing with linear constraints. Despite the task hierarchy, the introduction of a generalized task projector per task allows to solve only one LQP. This can be done by introducing as many virtual joint space variables as the number of tasks and using the generalized projector of each task in the expression of the constraints. The resulting problem can be solved by standard convex optimization tools and the cost of introducing virtual joint space variables is compensated for by the fact that only one optimization problem has to be solved. Details regarding this work are provided in \cite{liu2013}.

In the meantime, TUD has proposed to explore optimization methods for solving local control problems that do not require the explicit inversion of any model of the system.  During year one, TUD investigated Bayesian optimization methods that were applied to bipedal locomotion tasks.  One of the key challenges in robotic bipedal locomotion is finding gait parameters that optimize a desired performance criterion, such as speed, robustness or energy efficiency. Typically, gait optimization requires extensive robot experiments and specific expert knowledge. During year one, TUD demonstrated that  data-driven machine learning methods based on Bayesian optimization can be used to automate and speed up the process of gait optimization. These Bayesian optimization methods were used to efficiently find gait parameters that optimize the desired performance metric on a real bipedal walker \cite{calandra2014} and \cite{calandra2014b}.
    

\subparagraph{Bootstrapping and validating the control approach in rigid world and compliant cases (T3.4)}

During year one, UPMC has explored the contribution of MPC approaches to handle the postural balancing problem under varying contact conditions. The hybrid nature of the problem, where varying contact conditions can be accommodated either by adapting the internal forces distribution given a set of contact or by modifying the set of contacts itself, requires control approaches where the desired task trajectories performed through the local, reactive, whole-body controller have to be optimally planned ahead of time in order to provide robust behaviors. The contributions in this domain are mostly related to the work of A. Ibanez \cite{ibanez2013}, \cite{ibanez2014-icra} and \cite{ibanez2014-ark}. The originality of theses contributions lies in:
\begin{itemize}
	\item an augmented ZMP model including external forces exerted directly on indirectly on the center of mass;
	\item a distributed optimization approach that provides a way of generating reference trajectories for the center of mass representing a good compromise given some antagonistic balance and task;
	\item a non scripted foot step placement optimization.
\end{itemize}

UPMC also partially contributed with an experimental study about human behavior during physical contact with the robot. The protocol was registered and obtained the approval of the ethics committee CERES (Conseil d’\'evaluation \'ethique pour les recherches en sant\'e), from University Paris-Decartes. \footnote{Ivaldi et al., "Engagement during human-humanoid interaction", IRB N. 20135200001072.}
The purpose of the experiments is to collect a database of behaviors of experts and naive people interacting physically with the iCub to accomplish a cooperative task. The collected data also include locations of contacts (retrieved through the tactile skin), applied force, robot mouvements: they will be used to study adaptation to human intention during human-robot physical contacts.
    
In the meantime, TUD investigated the interchange of forces during cooperative tasks between humans and robots \cite{berger2013}. Three example scenarios are illustrated in Figure \ref{fig:interaction_tasks}. In such tasks, typically an exchange of forces takes place whenever the interacting agents make contact. Sometimes, forces are even exchanged through an object that is manipulated by both agents, e.g., through a box that is lifted. For a successful execution of such joint physical activities, a robot needs to accommodate for the external forces exerted by a human. To this end, we developed a machine learning approach for identifying external influences and guidance information from humans. During behavior execution by a robot, predictions from a statistical sensor model are continuously compared with stability parameters derived from current sensor readings. Differences between predicted and measured values exceeding the variance of the statistical model are interpreted as perturbations caused by a human and are used to adapt the robot's behavior.
    
\begin{figure}[!ht]
\centering
\includegraphics[width=\linewidth]{./images/interaction_wp3.png}
%\label{fig:subfig2}
 \caption{Illustration of three human robot interaction tasks, where an exchange of forces takes place investigated by TUD during year one.
}
\label{fig:interaction_tasks}
\end{figure}

\subparagraph{Deviations from workplan}  

The PM expenses for WP3 after one year of project are globally conform to the planned one. The observed deviations are related to the fact that tasks 3.3 and 3.4 spans the overall duration of the project and the contribution of some of the partners are expected in the 2nd, 3rd and 4th year.

%\emph{\color{red}[For work package 3 (UPMC) provide the following information:]}
%\begin{itemize}
%\item[-] \emph{\color{red}[A summary of progress towards objectives and details for each task;]}
%\item[-] \emph{\color{red}[Highlight clearly significant results;]}
%\item[-] \emph{\color{red}[If applicable, explain the reasons for deviations from Annex I and their impact on other tasks as well as on available resources and planning;]}
%\item[-] \emph{\color{red}[If applicable, explain the reasons for failing to achieve critical objectives and/or not being on schedule and explain the impact on other tasks as well as on available resources and planning (the explanations should be consistent with the declaration by the project coordinator) ;]}
%\item[-] \emph{\color{red}[a statement on the use of resources, in particular highlighting and explaining deviations between actual and planned  person-months per work package and per beneficiary in Annex 1 (Description of Work);]}
%\item[-] \emph{\color{red}[If applicable, propose corrective actions.]}
%\end{itemize}

\paragraph{Work package 4 progress}

\subparagraph{Generalizing and Improving Elementary Tasks with Contacts (T4.3)}

In this task, we aim to generate new skills from data, where elementary skills 
are acquired by imitation learning and transferred to novel situations using 
dynamic systems. During year one, TUD developed a novel representation of 
movement primitives that can be used for imitation learning from noisy observations.
Uncertainty of observed trajectories is explicitely modeled and used to generate new skills.
This movement representation has state-of-the-art capabilities in generalization, 
coupling between the degrees of freedom of the robot, and moreover, 
a time varying feedback controller can be derived in closed form. 
These features are partially illustrated in Figure \ref{fig:promps}.
This work was published
last year at the highly competitive conference on neural information processing \cite{Paraschos_NIPS_2013}.

\begin{figure}[!ht]
\centering
\includegraphics[width=\linewidth]{./images/ProMPs.png}
%\label{fig:subfig2}
 \caption{(a)
Conditioning on different target states. The blue shaded area represents the learned
trajectory distribution. We condition on different target positions, indicated by the ‘x’-markers. The
produced trajectories exactly reach the desired targets while keeping the shape of the demonstrations.
(b)
Combination of two ProMPs. The trajectory distributions are indicated by the blue and red
shaded areas. Both primitives have to reach via-points at different points in time, indicated by
the $x$��-markers. We co-activate both primitives with the same activation factor. The trajectory
distribution generated by the resulting feedback controller now goes through all four via-points.
(c)
Blending of two ProMPs. We smoothly blend from the red primitive to the blue primitive. The
activation factors are shown in the bottom. The resulting movement (green) first follows the red
primitive and, subsequently, switches to following the blue primitive.
}
\label{fig:promps}
\end{figure}

In another work, published at the international conference on humanoid robots (HUMANOIDS), 
TUD demonstrated that this probabilistic approach for trajectory generation
has superior performance against deterministic policies. The use of
probability distributions over the trajectories increased significantly
 the generalization properties, which was evaluated on a high dimensional table
tennis scenario [Paraschos, A. and  Neumann, G and  Peters, J., 2013]. 
In the future work, we plan to incorporate external torque signals to initiate, 
maintain, and terminate contacts.

TUD also investigated how to learn human robot interaction through imitation. We presented a new approach to robot learning that allows anthropomorphic robots to learn a library of interaction skills from demonstration [H Ben Amor, D Vogt, M Ewerton, E Berger, B Jung and J Peters, 2014]. Traditional approaches to modeling interactions assume a pre-specified symbolic representation of the available actions. For example, they model interactions
in terms of commands such as \emph{wait}, \emph{pick-up}, and \emph{place}. Instead of such a top-down approach, we focused on learning responsive behavior in a bottom-up fashion using a trajectory based approach. The key idea behind our approach is that the observation of human-human collaborations can provide rich information specifying how and when to interact
in a particular situation. For example, by observing how two human workmen collaborate on lifting a heavy box, a robot could use machine learning algorithms to extract an
interaction model that specifies the states, movements, and situational responses of the involved parties. In turn, such a model can be used by the robot to assist in a similar lifting task. Our approach is as an extension of imitation learning to multi-agent scenarios, in which the behavior and the mutual interplay between two agents is imitated

\begin{figure}[!ht]
\centering
\includegraphics[width=\linewidth]{./images/newoverview.png}
%\label{fig:subfig2}
 \caption{Illustration of the developed interaction primitives that allows to infer the behavior of the interacting partner.
}
\label{fig:interaction_primitives}
\end{figure}

We further extended the above approach by introducing \emph{Interaction Primitives} in [Ben Amor, H.; Neumann, G.; Kamthe, S.; Kroemer, O.; Peters, J., 2014]. Interaction primitives build on the framework of dynamic motor primitives (DMPs) by maintaining a distribution over the parameters of the DMP. With this distribution, we can learn the inherent correlations of cooperative activities which allow us to infer the behavior of the partner and to participate in the cooperation. A conceptual overview is sketched in Figure \ref{fig:interaction_primitives}. A learned Interaction Primitive can be used by a robot to (1) predict the human's next action in the current context, (2) identify the optimal response, (3) synchronize the movement with the human partner.

In the meantime, demonstration-based learning of "optimal trajectories" and stable controllers has been addressed by UPMC, in particular in \cite{stulp2013} where a general, flexible, and compact representation of parameterizable skills is proposed. This work generalizes the standard Dynamic Motor Primitive formulation in \cite{ijspeert2013} and proposes a novel DMP formulation for parametrized skills, based on additionally passing task parameters to the DMP function approximator. This generalizes previous approaches, in particular those which train and execute parametrized skills with two separate regressions. Learning the function approximator with one regression in the full space of phase and tasks parameters allows for more compact models, and the flexible use of different function approximator implementations such as LWPR and GPR, as we demonstrated on the Meka and iCub humanoids robots.

\paragraph{Work package 5 progress}

The activities in WP5 are divided into four tasks corresponding to the four years project duration. As a result, during the first year CoDyCo results concentrate on T5.1. The main result consist in the implementation of the validation scenario consisting of the balancing on different type of rigid contacts.

\subparagraph{Scenario 1: iCub balancing on multiple rigid contacts (T5.1)}

The main contributions to T5.1 have been presented in ``D5.1 Scientific report on validation scenario 1: balancing on multiple rigid contact points.'' which discusses the technical implementation of the first year validation scenario (see \url{https://github.com/robotology-playground/codyco-deliverables/tree/master/D5.1/pdf}). The software developed for the scenario implementation is released with an open-source license and distributed through github (\url{https://github.com/robotology/codyco} ). The main software activities include: a module to identify the whole-body motor transfer functions (\url{https://github.com/robotology/codyco/tree/master/src/modules/motorFrictionIdentification}), a module for estimating whole-body internal (joint torques) and external (contact) forces (\url{https://github.com/robotology/codyco/tree/master/src/modules/motorFrictionIdentification}), a module for whole-body joint torque control (\url{https://github.com/robotology/codyco/tree/master/src/modules/jointTorqueControl}), a C++ library that implements the wholeBodyInterface in simulink (\url{https://github.com/robotology/codyco/tree/master/src/simulink}).

\subparagraph{Deviations from workplan}  

The original work plan have foreseen contacts at feet, hands, back, buttocks, arms and legs. The final validation scenario will only include possible contacts at hands and feet. This simplification is mainly due to the fact that at he end of the CoDyCo first year the iCub does not yet include tactile sensing on the back, legs and buttocks. These sensors will be soon included in the iCub and the CoDyCo software is already designed to include this information. 

%\begin{itemize}
%\item[-] \emph{\color{red}[A summary of progress towards objectives and details for each task;]}
%\item[-] \emph{\color{red}[Highlight clearly significant results;]}
%\item[-] \emph{\color{red}[If applicable, explain the reasons for deviations from Annex I and their impact on other tasks as well as on available resources and planning;]}
%\item[-] \emph{\color{red}[If applicable, explain the reasons for failing to achieve critical objectives and/or not being on schedule and explain the impact on other tasks as well as on available resources and planning (the explanations should be consistent with the declaration by the project coordinator) ;]}
%\item[-] \emph{\color{red}[a statement on the use of resources, in particular highlighting and explaining deviations between actual and planned  person-months per work package and per beneficiary in Annex 1 (Description of Work);]}
%\item[-] \emph{\color{red}[If applicable, propose corrective actions.]}
%\end{itemize}


\paragraph{Work package 6 progress}

Activities within work package 6 achieved the expected results both in terms of administrative activities and management activities. As a major result, the software repository was successfully implemented thanks to novel versioning tool (git) and social coding website (\url{https://github.com}).

\subparagraph{Administrative coordination (T6.1)}
Administration was successfully coordinated by Chiara Andreoli at IIT. The major activity concerned an amendment that the CoDyCo consortium asked the main reason being the fact that Serena Ivaldi, initially hired by UPMC was recently hired by TUD. Part of the administrative coordination activities were also conducted during three main meetings: the kick-off meeting (Genoa, April 5th, 2013), the simulators meeting (Paris, June 5th, 2013) and the midyear meeting (Paris, November 21st-22nd, 2013). Details on the meetings can be found in the CoDyCo website (\url{http://www.codyco.eu}).

\subparagraph{Software repository implementation (T6.2)}

A github software repository was set up \url{https://github.com/robotology/codyco} and the contribution from the different developers can be directly checked in the website. Relevant information can be found also in ``D6.1 Website and repository online'' available here: \url{https://github.com/robotology-playground/codyco-deliverables/tree/master/D6.1/pdf}.


%\begin{itemize}
%\item[-] \emph{\color{red}[A summary of progress towards objectives and details for each task;]}
%\item[-] \emph{\color{red}[Highlight clearly significant results;]}
%\item[-] \emph{\color{red}[If applicable, explain the reasons for deviations from Annex I and their impact on other tasks as well as on available resources and planning;]}
%\item[-] \emph{\color{red}[If applicable, explain the reasons for failing to achieve critical objectives and/or not being on schedule and explain the impact on other tasks as well as on available resources and planning (the explanations should be consistent with the declaration by the project coordinator) ;]}
%\item[-] \emph{\color{red}[a statement on the use of resources, in particular highlighting and explaining deviations between actual and planned  person-months per work package and per beneficiary in Annex 1 (Description of Work);]}
%\item[-] \emph{\color{red}[If applicable, propose corrective actions.]}
%\end{itemize}

\paragraph{Work package 7 progress}

Dissemination and exploitation activities included the participation to international events addressed to both commercial and academic institutions. A preliminary exploitation plan was delineated and reported in the deliverable D7.1.

\subparagraph{Dissemination activities towards academia, industry, and other users (T7.1)}

Dissemination activities were conducted in three main events: (1) iCub exposition at ICRA 2013, IEEE International Conference on Robotics and Automation Karlsruhe, May 6 - 10, 2013; (2) iCub exposition at the European Robotics Forum and Innorobo, Lyon 29th March 2013; (3) iCub exposition at the European Robotics Forum, Rovereto 12th-14th of March 2014. The full list of papers published within CoDyCo can be found here: \url{http://codyco.eu/publications-menu}.

\subparagraph{Exploitation plan (T7.2)}

The first year activities on T7.1 and T7.2 are all contained in ``D7.1 Dissemination and exploitation plan'' available here: \url{https://github.com/robotology-playground/codyco-deliverables/tree/master/D7.1/pdf}.

\subparagraph{Management of IPR (T7.3)}

No activities to be reported during the first year on this task in consideration of the fact that the task started at the very end of the first year. As a minor starting activity the consortium circulated a list containing each partner responsible contact person for the IPR management. This list is contained in ``D7.1 Dissemination and exploitation plan'' available here: \url{https://github.com/robotology-playground/codyco-deliverables/tree/master/D7.1/pdf}.

\subparagraph{Dissemination of a database of human motion with contacts (T7.4)}

During the first year of CoDyCo, IIT completed the task of setting up a database for storing both human and robot datasets. The details on the database are reported in ``D7.2 Standard database with support materials'' available here \url{https://github.com/robotology-playground/codyco-deliverables/tree/master/D7.2/pdf}. 

\section{Second year results}

\input{sections/WP1/wp1_progress_recap.tex}
\input{sections/WP2/wp2_progress_recap.tex}
\input{sections/WP3/wp3_progress_recap.tex}
\input{sections/WP4/wp4_progress_recap.tex}
\input{sections/WP5/wp5_progress_recap.tex}
\input{sections/WP6/wp6_progress_recap.tex}
\input{sections/WP7/wp7_progress_recap.tex}

\input{sections/WP1/wp1_progress.tex}
%\input{sections/WP1/wp1_resources.tex}

\input{sections/WP2/wp2_progress.tex}
%\input{sections/WP2/wp2_resources.tex}

\input{sections/WP3/wp3_progress.tex}
%\input{sections/WP3/wp3_resources.tex}

\input{sections/WP4/wp4_progress.tex}
%\input{sections/WP4/wp4_resources.tex}

\input{sections/WP5/wp5_progress.tex}
%\input{sections/WP5/wp5_resources.tex}

\input{sections/WP6/wp6_progress.tex}
%\input{sections/WP6/wp6_resources.tex}

\input{sections/WP7/wp7_progress.tex}
%\input{sections/WP7/wp7_resources.tex}

\bibliography{ias_bibliography,ias_state_of_art}
\bibliographystyle{plain}

\end{document}
\endinput
%%
%% End of file `elsarticle-template-num.tex'.
